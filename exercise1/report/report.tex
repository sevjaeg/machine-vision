\documentclass[a4paper, 12pt]{article}

\usepackage[T1]{fontenc}
\usepackage[utf8]{inputenc}
\usepackage[english]{babel}

\usepackage{lmodern}  % nicer font
\usepackage{gensymb}
\usepackage{amsmath}
\usepackage{amssymb}
\usepackage{nicefrac}  % nicer inline fractions
\usepackage{listings}
\usepackage{enumerate}
\usepackage{siunitx}  % easy handling of value + unit (e.g. \SI{10}{\pF})
% \sisetup{}  % configure siunitx (e.g. locale = DE)
\usepackage{verbatim}
\usepackage{subcaption}  % captions for subplots
\usepackage[european, siunitx]{circuitikz}  % draw circuit diagrams
\usepackage{enumitem}
\setlist[itemize]{label=\rule[0.5ex]{0.6ex}{0.6ex}} % nice black squares for itemize environments
\usepackage{graphicx}
\graphicspath{{./figures/}}

\usepackage{geometry}
\geometry{%
	left   = 2.5cm,
	right  = 2.5cm,
	top    = 3cm,
	bottom = 3cm
}

\usepackage[  % ieee style citations (e.g. [1])
	backend     = biber,
	maxbibnames = 99,
	autocite    = footnote,
	style	    = ieee,
	citestyle   = numeric-comp,
	doi=false, isbn=false
]{biblatex}
\addbibresource{bibliography/bibliography.bib}

\usepackage[hang]{footmisc}
\renewcommand{\hangfootparindent}{2em} 
\renewcommand{\hangfootparskip}{2em}
\renewcommand{\footnotemargin}{0.00001pt}
\renewcommand{\footnotelayout}{\hspace{2em}}

% add missing hyphenations
\hyphenation{im-ple-men-ta-tions}

\title{376.054 Machine Vision and Cognitive Robotics\\
	   Exercise 1: Canny Edge Detector}
\author{
  Severin Jäger, 01613004
}
\date{\today}

% last imports! Modify Title and author
\usepackage[bookmarksopen,colorlinks,citecolor=black,linkcolor=black, urlcolor = black]{hyperref}
% after hyperref! Example: In \cref{label}, ... -> In section 2.1, ...
% works for all labels (figures, sections, chapters, equations, ...)
\usepackage[noabbrev, nameinlink]{cleveref} 


\begin{document}

\maketitle
\tableofcontents
\pagebreak

\section{Experiment 1}

\ref{image_lady}

\begin{figure} [h!]
\includegraphics[width=5cm]{ex1_lady/pos400.png}
\centering
\caption{row $400$ for \ref{1.2} highlighted}
\label{image_lady}
\end{figure}

\subsection{Kernel width as function of sigma}

\begin{equation}
W_{kernel} =  2 \cdot \textrm{round}(3 \sigma) + 1
\label{w_kernel}
\end{equation}

% no notable difference with small values of $\sigma$
\ref{kernel_sigma}

\begin{figure} [h!]
\begin{subfigure}[b]{0.3\textwidth}
\includegraphics[width=4.5cm]{ex1_lady/Blurred_Image_s=3_k=3.png}
\centering
\subcaption{$W_{kernel}=3$}
\end{subfigure}
\hfill
\begin{subfigure}[b]{0.3\textwidth}
\includegraphics[width=4.5cm]{ex1_lady/Blurred_Image_s=3_k=9.png}
\centering
\subcaption{$W_{kernel}=9$}
\end{subfigure}
\hfill
\begin{subfigure}[b]{0.3\textwidth}
\includegraphics[width=4.5cm]{ex1_lady/Blurred_Image_s=3.png}
\centering
\subcaption{$W_{kernel}=19$}
\end{subfigure}
\caption{Gaussian blurring with $\sigma = 3$ and different kernel sizes. The correct kernel width (as described in Eq. \ref{w_kernel}) is $19$.}
\label{kernel_sigma}
\end{figure}

\subsection{Effect of blurring on pixel intensities}
\label{1.2}

\ref{pixel_intensities}

\begin{figure} [h!]
\begin{subfigure}[b]{0.45\textwidth}
\includegraphics[width=8cm,trim={0 0 0 1.25cm},clip]{ex1_lady/intensities_400_gray.png}
\centering
\subcaption{Original Image}
\end{subfigure}
\hfill
\begin{subfigure}[b]{0.45\textwidth}
\includegraphics[width=8cm,trim={0 0 0 1.25cm},clip]{ex1_lady/intensities_400_blur1.png}
\centering
\subcaption{Blurred ($\sigma=1$)}
\end{subfigure}
\vskip\baselineskip
\begin{subfigure}[b]{0.45\textwidth}
\includegraphics[width=8cm,trim={0 0 0 1.25cm},clip]{ex1_lady/intensities_400_blur2.png}
\centering
\subcaption{Blurred ($\sigma=2$)}
\end{subfigure}
\hfill
\begin{subfigure}[b]{0.45\textwidth}
\includegraphics[width=8cm,trim={0 0 0 1.25cm},clip]{ex1_lady/intensities_400_blur3.png}
\centering
\subcaption{Blurred ($\sigma=3$)}
\end{subfigure}
\caption{Row $400$}
\label{pixel_intensities}
\end{figure}

\subsection{Effect of blurring on non-maximum suppression}


\ref{non_maximum}

\begin{figure} [h!]
\begin{subfigure}[b]{0.22\textwidth}
\includegraphics[width=4cm,trim={0 0 0.25cm 0},clip]{ex1_lady/Edges_Original_Image.png}
\centering
\subcaption{Original Image}
\end{subfigure}
\hfill
\begin{subfigure}[b]{0.22\textwidth}
\includegraphics[width=4cm,trim={0 0 0.25cm 0},clip]{ex1_lady/Edges_Blurred_Image_s=1.png}
\centering
\subcaption{Blurred ($\sigma=1$)}
\end{subfigure}
\hfill
\begin{subfigure}[b]{0.22\textwidth}
\includegraphics[width=4cm,trim={0 0 0.25cm 0},clip]{ex1_lady/Edges_Blurred_Image_s=2.png}
\centering
\subcaption{Blurred ($\sigma=2$)}
\end{subfigure}
\hfill
\begin{subfigure}[b]{0.22\textwidth}
\includegraphics[width=4cm,trim={0 0 0.25cm 0},clip]{ex1_lady/Edges_Blurred_Image_s=3.png}
\centering
\subcaption{Blurred ($\sigma=3$)}
\end{subfigure}
\caption{}
\label{non_maximum}
\end{figure}

\subsection{Effect of blurring on the edges after hysteresis thresholding}
	
\ref{hystersis}

$t_1 = 0.25, t_2 = 0.5$
\begin{figure} [h!]
\begin{subfigure}[b]{0.22\textwidth}
\includegraphics[width=4cm,trim={0 0 0.25cm 0},clip]{ex1_lady/Canny_Edges_Original.png}
\centering
\subcaption{Original Image}
\end{subfigure}
\hfill
\begin{subfigure}[b]{0.22\textwidth}
\includegraphics[width=4cm,trim={0 0 0.25cm 0},clip]{ex1_lady/Canny_Edges_Blurred_s=1.png}
\centering
\subcaption{Blurred ($\sigma=1$)}
\end{subfigure}
\hfill
\begin{subfigure}[b]{0.22\textwidth}
\includegraphics[width=4cm,trim={0 0 0.25cm 0},clip]{ex1_lady/Canny_Edges_Blurred_s=2.png}
\centering
\subcaption{Blurred ($\sigma=2$)}
\end{subfigure}
\hfill
\begin{subfigure}[b]{0.22\textwidth}
\includegraphics[width=4cm,trim={0 0 0.25cm 0},clip]{ex1_lady/Canny_Edges_Blurred_s=3.png}
\centering
\subcaption{Blurred ($\sigma=3$)}
\end{subfigure}
\caption{$t_1 = 0.25$ and $t_2 = 0.5$}
\label{hystersis}
\end{figure}

\clearpage
\section{Experiment 2}
\subsection{Parameters of hysteresis thresholding}

\subsection{Hysteresis thresholding on different images}

\subsection{Influence of automatic threshold selection}

\subsection{Limitations of automatic threshold selection}

\clearpage
\section{Experiment 3}
\subsection{Influence of noise on edge detection}

\subsection{Alleviating the influence of noise}

\sloppy
\printbibliography

\end{document}