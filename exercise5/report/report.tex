\documentclass[a4paper, 12pt]{article}

\usepackage[T1]{fontenc}
\usepackage[utf8]{inputenc}
\usepackage[english]{babel}

\usepackage{lmodern}  % nicer font
\usepackage{textcomp} % resolves warnings with gensymb
\usepackage{gensymb}
\usepackage{amsmath}
\usepackage{amssymb}
\usepackage{nicefrac}  % nicer inline fractions
\usepackage{listings}
\usepackage{enumerate}
\usepackage{booktabs}
\usepackage{siunitx}  % easy handling of value + unit (e.g. \SI{10}{\pF})
% \sisetup{}  % configure siunitx (e.g. locale = DE)
\usepackage{verbatim}
\usepackage{subcaption}  % captions for subplots
\usepackage[european, siunitx, RPvoltages]{circuitikz}  % draw circuit diagrams
\usepackage{enumitem}
\setlist[itemize]{label=\rule[0.5ex]{0.6ex}{0.6ex}} % nice black squares for itemize environments
\usepackage{graphicx}
\graphicspath{{./figures/}}

\usepackage{geometry}
\geometry{%
	left   = 2.5cm,
	right  = 2.5cm,
	top    = 3cm,
	bottom = 3cm
}

\usepackage[hang]{footmisc}
\renewcommand{\hangfootparindent}{2em} 
\renewcommand{\hangfootparskip}{2em}
\renewcommand{\footnotemargin}{0.00001pt}
\renewcommand{\footnotelayout}{\hspace{2em}}


\title{376.054 Machine Vision and Cognitive Robotics\\
	   Exercise 5: Plane fitting with RANSAC}
\author{
  Severin Jäger, 01613004
}
\date{\today}

% last imports! Modify Title and author
\usepackage[bookmarksopen,colorlinks,citecolor=black,linkcolor=black, urlcolor = black]{hyperref}
% after hyperref! Example: In \cref{label}, ... -> In section 2.1, ...
% works for all labels (figures, sections, chapters, equations, ...)
\usepackage[noabbrev, nameinlink]{cleveref} 

\begin{document}

\maketitle
\tableofcontents
\pagebreak

\section{Ground plane detection results}

In the first step, the basic RANSAC algorithm was used to detect the ground plane in an image as
shown in Figure \ref{ground_plane}. The plane was clearly detected, this is a reasonable start for segmenting the 
objects on top of the plane. However, this is a rather simple point cloud consisting basically of one plane and a few small 
objects atop.

\begin{figure}
	\centering
	\includegraphics[width=12cm]{basic_result.PNG}
	\caption{Ground plane detection result with a confidence of $0.9$, an inlier threshold of \SI{2}{\cm},
	 and a minimal sample distance of \SI{80}{\cm}}
	\label{ground_plane}
\end{figure}

\section{Influence of RANSAC parameters}

The RANSAC implementation used above has three major parameters: The inlier threshold which determines the maximum distance between 
the plane and points to be considered inliers, the minimal sample distance, which determines if a random sample is discarded due 
to the vicinity of the points and the confidence, which influences the termination of the RANSAC algorithm.

Figure \ref{high_inlier_t} (a) shows exemplarily a too high value of the inlier threshold. As a result, major parts of the objects are 
detected as part of the ground plane. This reduces the probability of correct segmentation an classification of the objects. However, 
a too small value can lead to significant problems as well. As shown in Figure \ref{high_inlier_t} (b), this yields planes not containing 
all relevant points due to sensor noise.

The minimal sample distance has less influence on the quality of the detected plane. In rare cases, it is possible that very close points 
lead to wrong planes, however the algorithm will find better results in later iterations.

Tuning the confidence does not only affect the quality of the fitted plane, but also the runtime of the algorithm. On one hand, a sufficient 
level is confidence is required for proper detection (s. Figure \ref{low_confidence}), on the other hand high values increase the number 
of iterations and thus the runtime. Therefore, this parameter involves a tradeoff between accuracy and complexity.

\begin{figure}
	\begin{subfigure}[b]{0.49\textwidth}
		\includegraphics[width=7.5cm]{high_5cm.PNG}
		\centering
		\subcaption{}
	\end{subfigure}
	\begin{subfigure}[b]{0.49\textwidth}
		\includegraphics[width=7.5cm]{low_inlier_t_2mm.PNG}
		\centering
		\subcaption{}
	\end{subfigure}
	\centering
	
	\caption{Ground plane detection with the parameters used for Figure \ref{ground_plane}, but an inlier threshold of \SI{5}{\cm} (a) and \SI{2}{\mm} (b).}
	\label{high_inlier_t}
\end{figure}

\begin{figure}
	\centering
	\includegraphics[width=12cm]{low_confidence.PNG}
	\caption{Ground plane detection with the parameters used for Figure \ref{ground_plane}, but a confidence of $0.12$}
	\label{low_confidence}
\end{figure}

\clearpage
\section{Multi-plane detection results}

\ref{door_basic}

\ref{desk_basic}

\ref{kitchen_basic}

\begin{figure}
	\centering
	\includegraphics[width=12cm]{door_basic.PNG}
	\caption{with a confidence of $0.9$, an inlier threshold of \SI{5}{\cm},
	and a minimal sample distance of \SI{30}{\cm}}
	\label{door_basic}
\end{figure}

\begin{figure}
	\centering
	\includegraphics[width=12cm]{desk_basic.PNG}
	\caption{with a confidence of $0.9$, an inlier threshold of \SI{5}{\cm},
	and a minimal sample distance of \SI{30}{\cm}}
	\label{desk_basic}
\end{figure}

\begin{figure}
	\centering
	\includegraphics[width=12cm]{kitchen_basic.PNG}
	\caption{with a confidence of $0.9$, an inlier threshold of \SI{5}{\cm},
	and a minimal sample distance of \SI{30}{\cm}}
	\label{kitchen_basic}
\end{figure}

% down-sampling approaches (all previous: uniform down-sampling)

\section{MSAC and MLESAC}


\end{document}
